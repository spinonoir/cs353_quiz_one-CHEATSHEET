\documentclass[10pt]{article}
\usepackage[margin=0.5in]{geometry}
\usepackage{amsmath}
\usepackage{graphicx}
\usepackage{multicol}
\setlength{\parindent}{0pt}
\pagestyle{empty}

\begin{document}

% --------------------------------------------------------------------
% Network Architecture and Layers
% --------------------------------------------------------------------
\section*{Network Architecture and Layers}

\textbf{OSI Model Layers:}
\begin{enumerate}
    \item \textbf{Physical Layer:} Concerns the transmission and reception of unstructured raw data over a physical medium. It includes hardware like cables, switches, and radio frequencies.
    \item \textbf{Data Link Layer:} Provides node-to-node data transfer—a link between two directly connected nodes. It handles error detection, frame synchronization, and flow control.
    \item \textbf{Network Layer:} Manages device addressing, tracks the location of devices on the network, and determines the best way to move data. Routers operate at this layer.
    \item \textbf{Transport Layer:} Ensures complete data transfer with error recovery and flow control. Protocols like TCP and UDP function here.
    \item \textbf{Session Layer:} Controls dialogues (sessions) between computers. It establishes, manages, and terminates the connections.
    \item \textbf{Presentation Layer:} Translates data between the application layer and the network format. It handles data encryption and decryption.
    \item \textbf{Application Layer:} Provides network services to the end-user applications. Protocols include HTTP, FTP, and SMTP.
\end{enumerate}

\textbf{Encapsulation Process:}
\begin{itemize}
    \item Data is encapsulated with protocol headers as it moves down the layers.
    \item Each layer adds its own header (and sometimes trailer) to the data from the layer above.
\end{itemize}

\textbf{Encapsulation Steps:}
\[
\begin{aligned}
\text{Application Data} & \rightarrow \text{Segment (Transport Header added)} \\
& \rightarrow \text{Packet (Network Header added)} \\
& \rightarrow \text{Frame (Data Link Header and Trailer added)} \\
& \rightarrow \text{Bits (Physical Layer)}
\end{aligned}
\]

% --------------------------------------------------------------------
% Key Protocols (HTTP, DNS, TCP/UDP, IP)
% --------------------------------------------------------------------
\section*{Key Protocols}

\textbf{HTTP (HyperText Transfer Protocol):}
\begin{itemize}
    \item Foundation of data communication for the World Wide Web.
    \item Stateless protocol operating at the application layer.
    \item Uses methods like GET and POST to request and submit data.
\end{itemize}

\textbf{DNS (Domain Name System):}
\begin{itemize}
    \item Translates human-friendly domain names to IP addresses.
    \item Hierarchical and decentralized naming system.
    \item Utilizes UDP port 53 for queries.
\end{itemize}

\textbf{TCP (Transmission Control Protocol):}
\begin{itemize}
    \item Ensures reliable, ordered, and error-checked delivery of data.
    \item Connection-oriented protocol with congestion control and flow control.
    \item Uses a three-way handshake to establish a connection.
\end{itemize}

\textbf{UDP (User Datagram Protocol):}
\begin{itemize}
    \item Provides a connectionless datagram service.
    \item Offers minimal error checking and no flow control.
    \item Suitable for applications that require speed over reliability (e.g., streaming).
\end{itemize}

\textbf{IP (Internet Protocol):}
\begin{itemize}
    \item Responsible for routing packets across network boundaries.
    \item Defines the addressing methods and structures of datagrams.
    \item IPv4 uses 32-bit addresses; IPv6 uses 128-bit addresses for expansion.
\end{itemize}

% --------------------------------------------------------------------
% Packet vs. Circuit Switching
% --------------------------------------------------------------------
\section*{Packet vs. Circuit Switching}

\textbf{Packet Switching:}
\begin{itemize}
    \item Data is broken into packets, each of which may take different paths to the destination.
    \item Network resources are used on-demand.
    \item Advantages: Efficient use of bandwidth, robustness.
    \item Disadvantages: Possible delays due to congestion, packet loss.
\end{itemize}

\textbf{Circuit Switching:}
\begin{itemize}
    \item A dedicated communication path is established between two nodes.
    \item Resources are reserved for the entire duration of the connection.
    \item Advantages: Continuous transmission with minimal delay.
    \item Disadvantages: Inefficient utilization of network resources.
\end{itemize}

\textbf{Visual Representation:}
\begin{itemize}
    \item \textbf{Packet Switching Diagram:}
    \[
    \text{[Host]} \rightarrow \text{[Router]} \rightarrow \text{[Router]} \rightarrow \text{[Host]}
    \]
    \item \textbf{Circuit Switching Diagram:}
    \[
    \text{[Host]} \leftrightarrow \text{[Router]} \leftrightarrow \text{[Router]} \leftrightarrow \text{[Host]}
    \]
\end{itemize}

% --------------------------------------------------------------------
% Performance Metrics (Delay, Throughput, Packet Loss)
% --------------------------------------------------------------------
\section*{Performance Metrics}

\textbf{Types of Delays:}
\begin{enumerate}
    \item \textbf{Processing Delay (\( d_{\text{proc}} \)):} Time routers take to process the packet header.
    \item \textbf{Queuing Delay (\( d_{\text{queue}} \)):} Time a packet spends in routing queues.
    \item \textbf{Transmission Delay (\( d_{\text{trans}} \)):} Time to push all packet bits onto the link.
    
    \[
    d_{\text{trans}} = \frac{\text{Packet Length (bits)}}{\text{Transmission Rate (bps)}}
    \]
    \item \textbf{Propagation Delay (\( d_{\text{prop}} \)):} Time for a signal to propagate through the medium.

    \[
    d_{\text{prop}} = \frac{\text{Distance (meters)}}{\text{Propagation Speed (meters/sec)}}
    \]
\end{enumerate}

\textbf{Total Delay:}
\[
\text{Total Delay} = d_{\text{proc}} + d_{\text{queue}} + d_{\text{trans}} + d_{\text{prop}}
\]

\textbf{Throughput:}
\begin{itemize}
    \item Rate at which bits are transferred between sender and receiver.
    \item \textbf{Instantaneous Throughput:} Measured at a given instant.
    \item \textbf{Average Throughput:} Measured over a longer period.
\end{itemize}

\textbf{Packet Loss:}
\begin{itemize}
    \item Occurs when network congestion leads to packets being dropped.
    \item Affects the quality of communication, requiring retransmissions.
\end{itemize}

% --------------------------------------------------------------------
% Web Technologies and CDNs
% --------------------------------------------------------------------
\section*{Web Technologies and CDNs}

\textbf{Web Technologies:}
\begin{itemize}
    \item \textbf{HTML (HyperText Markup Language):} Standard language for creating web pages.
    \item \textbf{CSS (Cascading Style Sheets):} Used to style and layout web pages.
    \item \textbf{JavaScript:} Scripting language for creating dynamic web content.
    \item \textbf{HTTPS:} Secure version of HTTP using SSL/TLS encryption.
    \item \textbf{Client-Server Model:} Clients request resources/services from servers.
\end{itemize}

\textbf{Content Delivery Networks (CDNs):}
\begin{itemize}
    \item Networks of distributed servers that deliver web content based on user location.
    \item Aim to reduce latency and improve page load times.
    \item Provide redundancy and reduce bandwidth costs.
\end{itemize}

\textbf{CDN Diagram:}
% \begin{center}
% \includegraphics[width=0.5\textwidth]{cdn_diagram.png}
% \end{center}

% --------------------------------------------------------------------
% Video Streaming Concepts
% --------------------------------------------------------------------
\section*{Video Streaming Concepts}

\textbf{Types of Streaming:}
\begin{itemize}
    \item \textbf{Live Streaming:} Broadcasting events in real-time.
    \item \textbf{On-Demand Streaming:} Accessing pre-recorded content anytime.
\end{itemize}

\textbf{Important Metrics:}
\begin{itemize}
    \item \textbf{Bandwidth:} Capacity of data transfer in a network.
    \item \textbf{Buffering:} Temporary storage to ensure smooth playback.
    \item \textbf{Bitrate:} Number of bits processed per unit of time, affecting video quality.
\end{itemize}

\textbf{Bitrate Calculation:}
\[
\text{Total Size (bits)} = \text{Bitrate (bps)} \times \text{Duration (sec)}
\]

\textbf{Example Calculation:}
\[
\begin{aligned}
\text{Bitrate} &= 2 \text{ Mbps} = 2 \times 10^6 \text{ bps} \\
\text{Duration} &= 60 \text{ sec} \\
\text{Total Size} &= 2 \times 10^6 \times 60 = 120 \times 10^6 \text{ bits} \\
&= 15 \text{ MB (since } 8 \times 10^6 \text{ bits} = 1 \text{ MB)}
\end{aligned}
\]

\end{document}